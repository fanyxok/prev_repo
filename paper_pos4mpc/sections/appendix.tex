\section{Appendix}\label{sec:appendix}

\subsection{MPC Protocols}\label{sec:protocols}
In general, there are three fundamental protocols to achieve semi-honest security:
garbled circuits~\cite{yao82,freexor,BeaverMR90}, secret sharing~\cite{Shamir79,GMW}, and homomorphic encryption~\cite{SHE},
varying in their computational, circuit and communication complexity.

\smallskip
\noindent
{\bf Garbled circuits}. The garbled circuit protocol was proposed by Yao for 2-party MPC.
Roughly speaking,
one party (say $\Party{1}$) encodes
the computation $f(x_1,x_2)$ as a Boolean circuit $C$, then garbles it.
The garbling procedure uses two random keys to denote the truth values of each wire
and symmetric-key encrypts the circuit $C$, ensuring that learning one random key for each fan-in allows to decrypt exactly one output of fan-out for each gate.
Next, $\Party{1}$ sends the garbled circuit, random keys of the truth values of $\Party{1}$'s private inputs,
and all the random keys of $\Party{2}$'s private inputs
to $\Party{2}$. %For each private input of the participant $\Party{2}$,
$\Party{2}$ chooses the random keys of the truth values of her private inputs and evaluates the garbled circuit %$GC$ %using the random keys of the inputs
%of both participants
by performing symmetric-key decryption. % for every gate.
%Recall that using one random key per input allows $\Party{2}$ to decrypt exactly one row of its garbled
%truth-table.
After successfully decrypting all the gates, $\Party{2}$ obtains the result of the garbled circuit
and sends it back to $\Party{1}$, from which $\Party{1}$ can obtain the result $f(x_1,x_2)$ and reveals
it to $\Party{2}$. To be semi-honest secure,
%for each his/her private input,
$\Party{2}$ is enforced to choose only one random key per $\Party{2}$'s private input and does not know the another random key, moreover, $\Party{1}$ does not know which random key
was chosen by  $\Party{2}$.
%This requirement is achieved via oblivious transfer~\cite{EvenGL85}.
%During the execution of the protocol, both participants only send and receive random keys,
%except for the final result $f(x_1,x_2)$.
%Thus, it is semi-honest secure.
Yao's garbled circuit protocol has been extended to multi-party MPC
with a distributed garbling procedure~\cite{BeaverMR90}.


\smallskip
\noindent
{\bf Secret sharing}. For an MPC application $f(\vec{x}_n)$, an $(n,t)$-secret sharing scheme   splits the private input $x_i$ of each party $\Party{i}$
into $n$ shares using random values and sends $n-1$ shares to the other parties, one share per party.
Any $t-1$ shares of the private input $x_i$ reveal no information of the private input $x_i$.
The computation $f(\vec{x}_n)$  is transformed into a Boolean or arithmetic circuit $C$ according to the secret sharing scheme
such that each party evaluates the circuit $C$ using the received shares from the other parties.
The results of the circuit $C$ of all the parties form
the share of the desired result $f(\vec{x}_n)$, from which
each party can reconstruct the result $f(\vec{x}_n)$.
For instance, using the Boolean secret sharing scheme, $x_i$ can be split into
the shares $(x_i^1,\cdots,x_i^n)$, where $x_i^1,\cdots,x_i^{n-1}$ are uniformly sampled
random values, $x_i^n=x_i^1\oplus \cdots\oplus x_i^{n-1}$ is the exclude-or of $x_i^1,\cdots,x_i^{n-1}$.
Using the additive secret sharing scheme, an element $x_i$ of a finite field can be split into
the shares $(x_i^1,\cdots,x_i^n)$,  $x_i^1,\cdots,x_i^{n-1}$ are uniformly sampled
random values from a finite field, $x_i^n=x_i^1+ \cdots+ x_i^{n-1}$, where $+$ is the finite field addition operation.

\medskip
\noindent
{\bf Homomorphic encryption}.
Homomorphic encryption enables computations on ciphertexts to compute
an encrypted result that, when decrypted, is the result $f(\vec{x}_n)$.
For instance, to compute an MPC application $f(\vec{x}_n)$,
all the parties encrypt their private inputs using the same public key
and then send the encrypted inputs to an untrusted third party.
The untrusted third party computes the encrypted result of $f(\vec{x}_n)$
using the encrypted inputs and sends the result back to the parties.
Finally, each party can obtain the result $f(\vec{x}_n)$ by decrypting
the encrypted result.

%The above three fundamental protocols are the base protocols for today's MPC.

%
%
%Yao firstly introduced the MPC problem in the early 1980s \cite{EvansKR18}.
%In \cite{yao82}, Yao proposed Yao's garbled circuit protocol for solving the MPC problem.
%Goldreich et al. \cite{GMW} proposed an additive secret sharing approach to solve MPC.
%A decade ago, Damg{\aa}rd et al. \cite{SHE} proposed a new MPC approach based on %a somewhat
%homomorphic cryptosystems.
%The above three protocols and their ideas became the basis of research work in the field of MPC protocol design.
%Garbled circuits, secret sharing, and homomorphic encryption have become the base protocols for today's MPC implementations.
%Due to the unresolved performance issues of homomorphic encryption, real-world MPC implementations tend to choose either garbled circuit-based protocols or secret-sharing-based protocols.
%In the experiment section, we consider experiments on the implementation of these two protocols.



\subsection{The Operational Semantics of {\LANG} Programs}
\label{sec:semantics}
The operational semantics of {\LANG} programs is shown in Fig.~\ref{fig:semantics}.
\begin{figure}[t]
    \centering
\scalebox{0.9}{
    \begin{tabular}{cc}
%    \hline\specialrule{0em}{3pt}{3pt}
%%%%%%%%%%%%%%%%%%%%%%%%%%%%%%%%%%%%%%%%%%
    $\begin{array}{c}
     \hline
        \MPCAngle{x=e, \sigma} \rightarrow \MPCAngle{{\Pskip}, \sigma[x\mapsto  \sigma(e)]}
    \end{array}$ &
%%%%%%%%%%%%%%%%%%%%%%%%%%%%%%%%%%%%%%%%%%
        $\begin{array}{c}
      %  \MPCState{e}{\sigma}=v \\
     \hline
        \MPCAngle{{\Preturn} \ x, \sigma} \rightarrow \MPCAngle{{\Pskip}, \sigma}
    \end{array}$ \\ \\
%%%%%%%%%%%%%%%%%%%%%%%%%%%%%%%%%%%%%%%%%%
   $\begin{array}{c}
        \MPCAngle{p_{1}, \sigma_{1}}             \rightarrow      \MPCAngle{ {\Pskip}, \sigma_{2} } \quad
        \MPCAngle{p_{2}, \sigma_{2}}    \rightarrow   \MPCAngle{ p_{2}' , \sigma_{3} } \\
        \hline
        \MPCAngle{p_{1};p_{2}, \sigma_{1}}       \rightarrow   \MPCAngle{p_{2}', \sigma_{3}}
    \end{array}$ \qquad\qquad&
%%%%%%%%%%%%%%%%%%%%%%%%%%%%%%%%%%%%%%%%%%
    $\begin{array}{c}
        \MPCAngle{p_1,\sigma_{1}}    \rightarrow    \MPCAngle{ p_1', \sigma_{2}}    \quad     p_{1}' \neq {\Pskip} \\
        \hline
        \MPCAngle{p_{1} ; p_{2}, \sigma_{1}} \rightarrow \MPCAngle{p_1^{\prime};p_2, \sigma_{2}}
    \end{array}$  \\ \\
%%%%%%%%%%%%%%%%%%%%%%%%%%%%%%%%%%%%%%%%%%
 $\begin{array}{c}
        p= \sigma(x) \ ? \ p_1 : \ p_2 \\
        \hline
        \MPCAngle{{\Pif} \ x \ {\Pthen} \  p_1 \ {\Pelse} \ p_2, \sigma}   \rightarrow  \MPCAngle{p,\sigma}
    \end{array} $ &
%%%%%%%%%%%%%%%%%%%%%%%%%%%%%%%%%%%%%%%%%%
   $ \begin{array}{c}
        p'=\sigma(x) \ ? \ p;{\Pwhile} \ x \ \Pdo \ p \ : \ {\Pskip} \\
        \hline
        \MPCAngle{{\Pwhile} \ x \ \Pdo \ p, \sigma}\rightarrow  \MPCAngle{p',\sigma}
    \end{array}$ \\ \\
%%%%%%%%%%%%%%%%%%%%%%%%%%%%%%%%%%%%%%%%%%
\multicolumn{2}{c}{$\begin{array}{c}
      p'= ( n\geq 1) \ ? \ p;{\Prepeat} \ n-1 \ \Pdo \ p \ : \ {\Pskip} \\
        \hline
        \MPCAngle{{\Prepeat} \ n \ \Pdo \ p, \sigma}\rightarrow  \MPCAngle{p',\sigma}
    \end{array}$ }\\ 
    \specialrule{0em}{3pt}{3pt}\hline
    \end{tabular}}
    \caption{The operational semantics of {\LANG} programs}
    \label{fig:semantics}
    % instrumented semantics
\end{figure}

\subsection{Proof of Proposition~\ref{prop:NI2Sec}}
{\bf Proposition~\ref{prop:NI2Sec}.}
\textit{If $p$ is $\varrho$-noninterfering for a security policy $\varrho$, then $\varrho\models_p f(\vec{x})$.}
%$\varrho$ is secure for the MPC application $f(x_1,\cdots, x_n)$
%whose functionality is implemented in $p$.



\begin{proof}
Assume the program $p$ is $\varrho$-noninterfering.
Consider the adversary-chosen  private inputs $\vec{v}_k\in\Dom^k$ for a constant $k\geq 1$, a result $v\in\Rng$ and a state $\sigma\in\leak_{\tt iw}^p(v,\vec{v}_k)$.
Let $\MPCAngle{\widehat{p}_\varrho,\sigma}\Downarrow \sigma_1:v$ be the execution
of $p$ starting from the state $\sigma$.
It suffices to show that $\leak_{\tt iw}^p(v,\vec{v}_k)\subseteq\leak_{\tt rw}^{p,\varrho}(v,\sigma)$,
as $\leak_{\tt iw}^p(v,\vec{v}_k)\supseteq\leak_{\tt rw}^{p,\varrho}(v,\sigma)$ always holds.
We show that $\sigma'\in\leak_{\tt rw}^{p,\varrho}(v,\sigma)$ for every $\sigma'\in\leak_{\tt iw}^p(v,\vec{v}_k)$.

Fix a state $\sigma'\in\leak_{\tt iw}^p(v,\vec{v}_k)$.
Let $\MPCAngle{\widehat{p}_\varrho,\sigma'}\Downarrow \sigma_1':v'$
be the execution of $p$ starting from the state $\sigma'$.
From $\sigma'\in\leak_{\tt iw}^p(v,\vec{v}_k)$, we get that $v'=v$.
Since $p$ is $\varrho$-noninterfering, we get that $\MPCAngle{\widehat{p}_\varrho,\sigma}\Downarrow_\varrho^{\Pub} \sigma_1:v$ and $\MPCAngle{\widehat{p}_\varrho,\sigma'}\Downarrow_\varrho^{\Pub} \sigma_1':v$
are the same. Therefore, $\sigma'\in\leak_{\tt rw}^{p,\varrho}(v,\sigma)$.\qed
\end{proof}




\subsection{Proof of Proposition~\ref{prop:symbleakage2}}
{\bf Proposition~\ref{prop:symbleakage2}.}
\textit{For each pair of
states $\sigma,\sigma'\in\leak_{\tt iw}^p(v,\vec{v}_k)$ such that
$\sigma\models \phi\wedge e=v$ and $\sigma'\models \phi'\wedge e'=v$,
if $\Psi_x(t,t')$ is valid and $\RW_{\varrho,\sigma}^{\Pub}(t)$ and $\RW_{\varrho,\sigma'}^{\Pub}(t')$ are identical,
then $\RW_{\varrho',\sigma}^{\Pub}(t)$ and $\RW_{\varrho',\sigma'}^{\Pub}(t')$ are identical,
where $\varrho'=\varrho[x\mapsto \Pub]$.}


\begin{proof}
Consider a pair of 
states $\sigma,\sigma'\in\leak_{\tt iw}^p(v,\vec{v}_k)$ such that
$\sigma\models \phi\wedge e=v$ and $\sigma'\models \phi'\wedge e'=v$.
Suppose $\RW_{\varrho,\sigma}^{\Pub}(t)$ and $\RW_{\varrho,\sigma'}^{\Pub}(t')$ are identical.
%Let $p_1,\cdots,p_m$ be the subsequence of the statements that are ${\Pif} \ x \ {\Pthen} \  p' \ {\Pelse} \ p''$ in $t$ and $t'$.
%Let $e_1,\cdots,e_m$ (resp. $e_1',\cdots,e_m'$) be symbolic values of $x$
%when executing $p_1,\cdots,p_m$ in the symbolic execution $t$ (resp. $t'$).


Let $\Primed(\sigma')$ be the state such that for every private input variable $x_i$,
$\Primed(\sigma')(x_i')=\sigma'(x_i)$.
Then, the model $(\sigma,\Primed(\sigma'))$ satisfies the constraint $\phi\wedge \Primed(\phi')\wedge e=\Primed(e')$.
Since $\Psi_x(t,t')$ is valid, the model $(\sigma,\Primed(\sigma'))$ also satisfies the constraint $\big(\bigwedge_{i=1}^m e_i=\Primed(e_i')\big)$.
This implies that once the same conditional statement $p_i$ is executed in $\RW_{\varrho,\sigma}(t)$ and $\RW_{\varrho,\sigma'}(t')$, they enter the same branch.
Therefore, $\RW_{\varrho',\sigma}^{\Pub}(t)$ and $\RW_{\varrho',\sigma'}^{\Pub}(t')$ are identical.
\qed
\end{proof}

\subsection{Proof of Theorem~\ref{thm:mainresult}}
{\bf Theorem~\ref{thm:mainresult}.}
\textit{If $\varrho\models_p f(\vec{x})$ and $\Psi_x(t,t')$ is valid for each pair of symbolic executions $t,t'\in\SymExe$ %such that ${\tt Stmt}_{\widehat{p}_\varrho}(t)$
%and ${\tt Stmt}_{\widehat{p}_\varrho}(t')$ are identical,
then, $\varrho[x\mapsto \Pub]\models_p f(\vec{x})$.}


\begin{proof}
Suppose $\varrho\models_p f(\vec{x})$. Then $\Enc_\varrho(1,p)$ does not raise any errors,
 and for every adversary-chosen private inputs $\vec{v}_k\in\Dom^k$, result $v\in\Rng$,
and state $\sigma\in\leak_{\tt iw}^p(v,\vec{v}_k)$,
$\leak_{\tt iw}^p(v,\vec{v}_k)=\leak_{\tt rw}^{p,\varrho}(v,\sigma).$
We immediately get that $\Enc_{\varrho'}(1,p)$ does not raise any errors.
It suffices to show that $\leak_{\tt rw}^{p,\varrho}(v,\sigma)=\leak_{\tt rw}^{p,\varrho'}(v,\sigma)$, where
$\varrho'=\varrho[x\mapsto \Pub]$.
Since the policy $\varrho'$ releases no less information than $\varrho$ to the adversary, then
$\leak_{\tt rw}^{p,\varrho}(v,\sigma)\supseteq\leak_{\tt rw}^{p,\varrho'}(v,\sigma)$. % $\Sigma^{p,\varrho}_{v,\sigma}\supseteq\Sigma^{p,\varrho'}_{v,\sigma}$.
We show that $\leak_{\tt rw}^{p,\varrho}(v,\sigma)\subseteq\leak_{\tt rw}^{p,\varrho'}(v,\sigma)$.
The case where $|\leak_{\tt rw}^{p,\varrho}(v,\sigma)|\leq 1$ is trivial, as each concrete
execution has a corresponding symbolic execution. We consider the case where $|\leak_{\tt rw}^{p,\varrho}(v,\sigma)|\geq 2$.

Consider a pair of states $\sigma,\sigma'\in\leak_{\tt rw}^{p,\varrho}(v,\sigma)$.
Then, $\MPCAngle{\widehat{p}_{\varrho},\sigma}\Downarrow_{\varrho}^{\Pub} \sigma_1:v$
and $\MPCAngle{\widehat{p}_{\varrho},\sigma'}\Downarrow_{\varrho}^{\Pub} \sigma_1':v$ are identical.
There exist two symbolic executions $t=\SymMPCAngle{p, \alpha_0,\True}\Downarrow
(\alpha,\phi):e, \ t'=\SymMPCAngle{p, \alpha_0,\True}\Downarrow
(\alpha',\phi'):e'\in\SymExe$
such that $\RW_{\varrho,\sigma}^{\Pub}(t) = \MPCAngle{\widehat{p}_{\varrho},\sigma}\Downarrow_{\varrho}^{\Pub} \sigma_1:v$ and
$\RW_{\varrho,\sigma'}^{\Pub}(t')=\MPCAngle{\widehat{p}_{\varrho},\sigma'}\Downarrow_{\varrho}^{\Pub} \sigma_1':v$.
By Proposition~\ref{prop:symbleakage2}, we get that
 $\RW_{\varrho',\sigma}^{\Pub}(t)$ and $\RW_{\varrho',\sigma'}^{\Pub}(t')$ are identical.
Thus, $\sigma,\sigma'\in\leak_{\tt rw}^{p,\varrho'}(v,\sigma)$.
\qed
\end{proof}
