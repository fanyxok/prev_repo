\section{Related work}\label{sec:relatedwork}
% \subsection{Multi-party computation frameworks}
% Obliv-C\cite{oblivc}  is a simple GCC wrapper that makes embedding secure computing protocols in regular C programs. Obliv-C implements Yao's GC protocol and provides programmers a flexible control to the program executions. The Obliv-C compiler transforms an Obliv-C language program into a secure multi-party encryption protocol. The protocol performs operations without revealing any input or computed intermediate values to either party. Only the final output will be displayed.

% MPyC\cite{mpyc} is a python library for MPC. It provides a python interface for fast prototyping and execution against the malicious adversary. MPyC implements Shamir's threshold secret sharing over finite fields, supports secure m-party computation.

% MP-SPDZ \cite{spdz20} features a python-based front end and implements a large set of schemes, including multi-party circuit-based, hybrid, and garbled circuit protocols.  MP-SPDZ implements MPC protocol variants that cover the most commonly used security models.  However, MP-SPDZ generates the whole circuit in the compilation process so that it can't dynamically generate circuits during the execution.

% ABY\cite{aby2}, EMP toolkit\cite{emp}, Frigate\cite{frigate16} and etc also perform secure computation. A recent SoK\cite{mpcsok} gives the survey and comparison of most MPC compilers.



\noindent
{\bf MPC Frameworks.}
Early efforts to MPC frameworks provide high-level languages for specifying MPC applications and compilers
for translating them into executable implementations~\cite{Fairplay04,Ben-DavidNP08,BogdanovLW08,DamgardGKN09,BurkhartSMD10,HolzerFKV12,SonghoriHS0K15,frigate16}.
For instance, Fairplay complies 2-party MPC programs written in a domain-specific language into
Yao's garbled circuits~\cite{Fairplay04}. FairplayMP~\cite{Ben-DavidNP08} extends
Fairplay to multi-party using a modified version of the BMR protocol~\cite{BeaverMR90} with a Java-language interface.
The others~\cite{BogdanovLW08,DamgardGKN09,BurkhartSMD10,HolzerFKV12,SonghoriHS0K15,frigate16}
are aimed to improve the efficiency of operations in circuits and size of circuits.
%Sharemind~\cite{BogdanovLW08}, VIFF~\cite{DamgardGKN09}, SEPIA~\cite{BurkhartSMD10}, CBMC-GC~\cite{HolzerFKV12}, TinyGarble~\cite{SonghoriHS0K15} and Frigate~\cite{frigate16}
%are proposed to improve the efficiency of operations in circuits and size of circuits.
Mixed MPC protocols were also proposed to improve efficiency~\cite{HeneckaKSSW10,SchropferKM11,BogdanovLR14,RastogiHH14,LaudR15,Demmler0Z15,MohasselR18,hycc18,aby221}, as the efficiency of MPC protocols vary in operations.
These frameworks explore the implementation space of operations in specific MPC protocols (e.g., garbled circuits, secret sharing
and homomorphic encryption), as well as their conversions.
%Among them, Wysteria~\cite{RastogiHH14}
%is the first work that formalizes operational semantics of its domain-specific language and proposes a refinement type system for reasoning about correctness.
However, all these frameworks either entirely compile an MPC program
or compile an MPC program according to user-annotated secret variables to improve performance, but without formal security guarantees.
Our approach improves the performance of MPC applications by declassifying secret variables without compromising security,
thus it is orthogonal to the above optimization works.


\smallskip
\noindent
{\bf Security of MPC applications.}
Since MPC applications implemented in MPC frameworks do not necessarily secure
due to information leakage during execution in the real-world. Therefore,
information-flow type systems and data flow analysis have been adopted in the MPC frameworks~\cite{NielsenS07,Mitchell0SZ12,picco13,LiuHSKH14,LiuWNHS15,ZahurE15,ChandranGRST19,MPyC20,spdz20}.
%EzPC~\cite{ChandranGRST19} SMCL~\cite{NielsenS07}, PICCO~\cite{picco13}, SCVM~\cite{LiuHSKH14}, ObliVM~\cite{LiuWNHS15}, Obliv-C~\cite{ZahurE15}, MPyC~\cite{MPyC20}, MP-SPDZ~\cite{spdz20},
%and the Haskell extension~\cite{Mitchell0SZ12}.
However, these works only consider security verification, instead of
automatically generation of security policies as we did.
Moreover,
these approaches cannot identify some variables (e.g., {\tt c2} in our motivating example)
that can be declassified without compromising security.
Kerschbaum~\cite{Kerschbaum11} proposed to infer public intermediate values
by reasoning about epistemic modal logic, similar goal to ours for declassifying secret variables.
However, it is unclear how efficient is this approach, as the performance of their approach was
not reported in~\cite{Kerschbaum11}.



Alternatively, self-composition  which reduces the security problem to
the safety problem on two copies of a program has been adopted by~\cite{RastogiMHH13,mpcleak18,RastogiSH19},
%\fan{\cite{,spdz20} are common frameworks,  didn't consider declassify variables and didn't provide program verification functionality. Remove them?}
where the safety problem can be solved by safety verification tools.
However, safety verification remains challenging and these approaches often require user annotations (e.g., procedure contracts and loop invariants) that are non-trivial for MPC practitioners. %and users
%often have to annotate loop invariants, while writing
%useful loop invariants is non-trivial for MPC developers.
Compared over these works,
there are three differences:
(1) they only use the self-composition reduction to verify security instead of automatically generating a security policy;
(2) they have to check almost all the program variables which is computational expensive,
 while we first apply an efficient type system to infer a security policy
and then only check if the security branching variables in the security policy
can be declassified;
and (3) we check if security branching variables can be declassified by reasoning about pairs of symbolic executions which can be seen as a divide-and-conquer approach
without annotations and the results are fed back to the type system to efficiently refine security levels.
We also remark that the self-composition reduction could also be used to check if a security branching variable
could be declassified.




%TASTY~\cite{HeneckaKSSW10} combines homomorphic encryption
%with Yao's garbled circuits. L1~\cite{SchropferKM11} proposes a language for specifying mixed protocols.
%Sharemind is extended mixed protocols~\cite{BogdanovLR14,LaudR15}.
% ABY~\cite{Demmler0Z15} ABY2.0~\cite{aby221}, ABY$^3$~\cite{MohasselR18}, HyCC~\cite{hycc18},

 %EMP-toolkit~\cite{emp-toolkit16},



\smallskip
\noindent
{\bf Information-flow analysis.}
A rich body of literature has studied the verification of information-flow security and noninterference
in programs~\cite{DenningD77,GoguenM82a}, which requires that confidential data does not flow to
outputs. This is too restrictive for programs which allows secret data to flow to some non-secret outputs, e.g.,
MPC applications, therefore the security notion is extended with declassification (a.k.a. delimited release) later~\cite{LiZ05,SabelfeldS09}.
These security problems are verified by type systems (e.g.~\cite{VolpanoIS96,SabelfeldM03,LiZ05}) or
self-composition (e.g.,~\cite{TerauchiA05,BartheDR11}) or relational reasoning~(e.g.,~\cite{AmtoftBB06,SousaD16,BartheCK11}).
Some of these techniques have been adapted to verify timing side-channel security, e.g.,~\cite{AlmeidaBBDE16,ChenFD17,YangVSGM18,CauligiSJBWRGBJ19}.
However, the usual notions of security in these settings do not require reasoning about arbitrary leakage
and these works are not directly applicable to our setting, although some of those techniques have been adapted to verify
security of MPC applications.




\begin{comment}

\subsection{Trade-off between performance and privacy}
Almeida et al. and  Fischer et al. study the tradeoff between performance and privacy for privacy-preserving computation in recent years.
Both work \cite{mpcleak18,pasapto} consider revealing information about private data to improve performance.
\cite{mpcleak18} allows a semantic leakage upper bound and uses Boogie verification toolchain to verify whether the leakage of a program is beyond the given upper bound.
The programmer gives the upper bound and writes annotations including precondition, postcondition, and loop invariants into source code to guide the theorem prover to complete the verification.
The verification conditions heavily rely on the knowledge of programmers.

\cite{pasapto} quantizes the information leakage of private inputs as the number of bits.
It models the tradeoff as an optimization problem.
\cite{pasapto} heuristically searches the variants which reveal some privacy information flow of the program to find an optimal variant within a user given leakage upper bound.

\subsection{Circuit level optimization}
Circuit level optimization reduces gate cost for general MPC.
The state-of-the-art circuit level optimization technique is Half Gates \cite{halfgate} which leverages prior optimization work \cite{BeaverMR90,grr3,freexor}.
Half Gates divides an AND gate into two half gates, one for garbler and the other one for evaluator.
Half Gates needs two ciphertexts in communications for an AND gate while the classical gates need eight ciphertexts.
Half Gates needs no communications for XOR gates that are compatible with Free XOR \cite{freexor}.

\subsection{Specified protocol}
For MPC problems with important applications, researchers always design specified protocols to achieve higher performance than general protocols.
For example, the Private Set Intersection (PSI) problem.

Pinkas et al. \cite{pssz} design a two-party PSI protocol using permutation-based hashing.
The performance of \cite{pssz} is much better than circuit-based PSI and OT-based PSI
\cite{pssz} achieves 3$\times$-4$\times$ overhead than insecure hashing PSI protocol for 32-bit item length and sufficiently large sets.

Kolesnikov et al. \cite{kol} proposed the state-of-the-art PSI protocol based on \cite{pssz} with an efficient OPRF.
\cite{kol} achieves about 3$\times$ speedup than \cite{pssz} for 128 bit item length and sufficiently large sets.

\subsection{Hybrid protocol}
ABY \cite{aby221} is a hybrid protocol framework for two-party MPC.
ABY provides switches among Arithmetic GMW protocol,  Boolean GMW protocol, and Yao's garbled circuit protocol.
Programmers can assign different protocols to different computations to achieve higher performance.
For example, computing arithmetic operations using Arithmetic GMW protocol and computing comparison operations using Yao's garbled circuit protocol.

HyCC \cite{hycc18} proposes the protocol selection for hybird protocol.
HyCC splits the full functionality into small parts.
In the protocol selection phase, HyCC benchmarks the cost of each operation and decides the protocol for each small part by optimizing a cost function.
ABY \cite{aby221} assigns protocol to operations during the development process.
The assignment relies on the knowledge of programmers.
But HyCC \cite{hycc18} selects protocol according to the running environment.
\end{comment}
