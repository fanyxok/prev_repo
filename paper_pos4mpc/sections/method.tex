\section{Degrading Security Levels}\label{sec:symbolicreasoning}
The type system allows to infer a security policy $\varrho$ such that the type judgement
$\Pub\vdash p:\varrho_0\Rightarrow\varrho$ is valid, from which we can deduce that
$\varrho\models_p f(\vec{x})$,
i.e., $\varrho$ is perfect for the MPC application $f(\vec{x})$ implemented by the program $p$.
However, the security policy $\varrho$ may be too conservative,
i.e., some secret variables specified by $\varrho$ can be declassified
without compromising the security. % so the computation and communication cost can be reduced.
In this section, we propose an automated approach to identify these variables. We mainly consider secret branching variables, viz., the secret variables used in branching conditions, as they usually
%It is because secret-dependent conditions should be removed via multiplexers, i.e., both branches of a secret-dependent conditional statement are executed whatever the truth value of the branching condition,  hence incurring
incur a high computation and communication overhead.
%Hereafter, a secret variable used as a branching condition will be called a secret condition \tl{why use this name?}.
%
W.l.o.g., we assume that for each secret branching variable  $x$ there is only one assignment to $x$ and it is used only in one conditional statement. %${\Pif} \ x \ {\Pthen} \  p_1 \ {\Pelse} \ p_2$.
(We can rename variables in $p$ if this assumption does not hold, where
the named variables have the same security levels as their original names.)
With this assumption, whether $x$ can be declassified depends
only on the unique conditional statement where it occurs.

Fix a security policy $\varrho$ such that $\varrho\models_p f(\vec{x})$.
%i.e., $\Enc_\varrho(1,p)$ does not raise any errors, and for every adversary-chosen private inputs $\vec{v}_k\in\Dom^k$, result $v\in\Rng$,
%and initial states $\sigma,\sigma'\in\leak_{\tt iw}^p(v,\vec{v}_k)$,
%$\MPCAngle{\widehat{p}_\varrho,\sigma'}\Downarrow_\varrho^{\Pub} \sigma_1':v$
%and $\MPCAngle{\widehat{p}_\varrho,\sigma}\Downarrow_\varrho^{\Pub} \sigma_1:v$ are identical.
Suppose that ${\Pif} \ x \ {\Pthen} \  p_1 \ {\Pelse} \ p_2$ is not used with secret-dependent conditions.
Let $\varrho'$ be the security policy $\varrho[x\mapsto \Pub]$. It is easy to see that
$\Enc_{\varrho'}(1,p)$ does not raise any errors.
Therefore, to declassify  $x$,
we need to ensure that %$\varrho'\models_p f(x_1,\cdots, x_n)$, i.e.,
$\MPCAngle{\widehat{p}_{\varrho'},\sigma'}\Downarrow_{\varrho'}^{\Pub} \sigma_1':v$
and $\MPCAngle{\widehat{p}_{\varrho'},\sigma}\Downarrow_{\varrho'}^{\Pub} \sigma_1:v$ are identical
for every adversary-chosen private inputs  $\vec{v}_k\in\Dom^k$, result $v\in\Rng$,
and states $\sigma,\sigma'\in\leak_{\tt iw}^p(v,\vec{v}_k)$.
However, %it is challenging,
as the number of the initial states may be large and even infinite,
it is infeasible to  check all pairs of executions.

We propose to use symbolic executions to represent the potentially infinite sets of (concrete) executions. %where
%one symbolic execution $t$ denotes a set of executions $C$.
Each symbolic execution $t$ is associated with a path condition $\phi$ which denotes the set of initial states satisfying $\phi$, from each of which the execution  has the same sequence of statements.
Thus, the conjunction $\phi\wedge e=v$, where $e$ is the symbolic return value and $v$ is concrete value,
%of the path condition $\phi$ with the symbolic return value $e$ being a concrete value $v$
represents the set of initial states from which the executions have the same sequence of statements
and returns the same result $v$. It is not difficult to observe that checking  whether $x$ in ${\Pif} \ x \ {\Pthen} \  p_1 \ {\Pelse} \ p_2$
can be declassified amounts to checking whether for every pair of symbolic executions
$t_1$ and $t_2$ that both include ${\Pif} \ x \ {\Pthen} \  p_1 \ {\Pelse} \ p_2$,
$x$ has the same truth value in $t_1$ and $t_2$ whenever $t_1$ and $t_2$ return the same value. This can be solved
by invoking off-the-shelf SMT solvers.

%In the rest of this section, we first introduce the symbolic semantics and
%then present our approach to automatically identity secret branching variables that can be declassified
%via reasoning about pairs of symbolic executions.
%Since a secret variable $x$ may be used in multiply statements, where some occurrences of $x$ can be publicable while
%the other occurrences of $x$ cannot be publicable,


\subsection{Symbolic Semantics}
Let $\Exp$ denote the set of expressions over the private input variables $\vec{x}$ and constants.
A \emph{path condition} $\phi\in\Exp$ is a conjunction of Boolean expressions.
A state $\sigma\in\InitState_0$ satisfies $\phi$, denoted by $\sigma\models \phi$, if
$\phi$ evaluates to $\True$ under $\sigma$. %when the private input variables $x_i$ are replaced by their values $\sigma(x_i)$.
A \emph{symbolic state} $\alpha$ is a function $\Var\rightarrow \Exp$ that maps variables to symbolic
expressions. $\alpha(e)$ denotes the symbolic value of the expression
$e$ under $\alpha$, obtained from $e$ by replacing each occurrence
of variable $x$ by $\alpha(x)$.
The initial symbolic state, denoted by $\alpha_0$,
is the identity function over the private input variables $\vec{x}$.


\begin{figure}[t]
    \centering
\scalebox{0.9}{	
\begin{tabular}{cc}
%\hline \specialrule{0em}{3pt}{3pt}
%%%%%%%%%%%%%%%%%%%%%%%%%%%%%%%%%%%%%%%%%%
    $\begin{array}{c}
        \hline
        \SymMPCAngle{x=e, \alpha,\phi} \hookrightarrow \SymMPCAngle{{\Pskip}, \alpha[x\mapsto \alpha(e),\phi]}
    \end{array}$ &
    %%%%%%%%%%%%%%%%%%%%%%%%%%%%%%%%%%%%%%%%%%
        $\begin{array}{c}
   \hline
        \SymMPCAngle{{\Preturn} \ x, \alpha,\phi} \hookrightarrow \SymMPCAngle{{\Pskip}, \alpha,\phi}
    \end{array}$ \\ \specialrule{0em}{3pt}{3pt}
%%%%%%%%%%%%%%%%%%%%%%%%%%%%%%%%%%%%%%%%%%
   $\begin{array}{c}
     \makecell[l]{\SymMPCAngle{p_{1}, \alpha_{1},\phi_1}             \hookrightarrow      \SymMPCAngle{ {\Pskip}, \alpha_{2}, \phi_2} \\
     \SymMPCAngle{p_{2}, \alpha_{2},\phi_2}    \hookrightarrow   \SymMPCAngle{ p_{2}' , \alpha_{3},\phi_3 }} \\
        \hline
     \SymMPCAngle{p_{1};p_{2}, \alpha_{1},\phi_1}       \hookrightarrow   \SymMPCAngle{p_{2}', \alpha_{3},\phi_3 }
    \end{array}$ &
%%%%%%%%%%%%%%%%%%%%%%%%%%%%%%%%%%%%%%%%%%
    $\begin{array}{c}
      \makecell[c]{\SymMPCAngle{p_1,\alpha_{1},\phi_1}    \hookrightarrow    \SymMPCAngle{ p_1', \alpha_{2},\phi_2}    \\     p_{1}' \neq {\Pskip}}\\
        \hline
      \SymMPCAngle{p_{1} ; p_{2}, \alpha_{1},\phi_1} \hookrightarrow \SymMPCAngle{p_1^{\prime};p_2, \alpha_{2},\phi_2}
    \end{array}$  \\ \specialrule{0em}{3pt}{3pt}
%%%%%%%%%%%%%%%%%%%%%%%%%%%%%%%%%%%%%%%%%%
 $\begin{array}{c}
      \SAT(\phi') \quad \phi'= \phi\wedge \alpha(x)\\
        \hline
     \SymMPCAngle{{\Pif} \ x \ {\Pthen} \  p_1 \ {\Pelse} \ p_2, \alpha,\phi}   \hookrightarrow  \SymMPCAngle{p_1,\alpha,\phi'}
    \end{array} $ \qquad\qquad&
    %%%%%%%%%%%%%%%%%%%%%%%%%%%%%%%%%%%%%%%%%%
 $\begin{array}{c}
      \SAT(\phi') \quad \phi'=\phi\wedge \neg\alpha(x)\\
        \hline
       \SymMPCAngle{{\Pif} \ x \ {\Pthen} \  p_1 \ {\Pelse} \ p_2, \alpha,\phi}   \hookrightarrow  \SymMPCAngle{p_2,\alpha,\phi'}
    \end{array} $ \\ \specialrule{0em}{3pt}{3pt}
%%%%%%%%%%%%%%%%%%%%%%%%%%%%%%%%%%%%%%%%%%
   $ \begin{array}{c}
        \SAT(\phi')\quad \phi'=\phi\wedge \alpha(x) \quad p'= p;{\Pwhile} \ x \ \Pdo \ p \\
        \hline
    \SymMPCAngle{{\Pwhile} \ x \ \Pdo \ p, \alpha,\phi}\hookrightarrow  \SymMPCAngle{p',\alpha,\phi'}
    \end{array}$ \qquad \qquad&
       $ \begin{array}{c}
          \SAT(\phi')\quad \phi'=\phi\wedge \neg\alpha(x)  \quad p'=\ {\Pskip} \\
        \hline
       \SymMPCAngle{{\Pwhile} \ x \ \Pdo \ p, \alpha,\phi}\hookrightarrow  \SymMPCAngle{p',\alpha,\phi'}
    \end{array}$  \\ \specialrule{0em}{3pt}{3pt}
%%%%%%%%%%%%%%%%%%%%%%%%%%%%%%%%%%%%%%%%%%
\multicolumn{2}{c}{$\begin{array}{c}
      p'= ( n\geq 1) \ ? \ p;{\Prepeat} \ n-1 \ \Pdo \ p \ : \ {\Pskip} \\
        \hline
        \SymMPCAngle{{\Prepeat} \ n \ \Pdo \ p, \alpha,\phi}\hookrightarrow  \SymMPCAngle{p',\alpha,\phi}
    \end{array}$  }\\
   \specialrule{0em}{3pt}{3pt}    \hline
    \end{tabular}}
\vspace{-2mm}
    \caption{The symbolic semantics of {\LANG} programs}
    \label{fig:symbsemantics}
\vspace{-3mm}
    % instrumented semantics
\end{figure}

The symbolic semantics of {\LANG} programs is defined by transitions between symbolic configurations, as shown in Fig.~\ref{fig:symbsemantics},
where $\SAT(\phi)$ is {\True} iff the constraint $\phi$ is satisfiable. %, otherwise {\False}.
A \emph{symbolic configuration} is a tuple $\SymMPCAngle{p, \alpha,\phi}$, where $p$ is a statement, $\alpha$ is a symbolic state,
and $\phi$ is the path condition that should be satisfied to reach $\SymMPCAngle{p, \alpha,\phi}$.
$\SymMPCAngle{p, \alpha,\phi}\hookrightarrow \SymMPCAngle{p', \alpha',\phi'}$ denotes a transition from %a symbolic configuration
$\SymMPCAngle{p, \alpha,\phi}$ to %a symbolic configuration
$\SymMPCAngle{p', \alpha',\phi'}$.
The symbolic semantics is almost the same as the operational semantics except that
(1) the path conditions are collected and checked for conditional statements and {\Pwhile} loops,
and (2) the transition may be non-deterministic if both $\phi\wedge \alpha(x)$
and $\phi\wedge \neg\alpha(x)$ are  satisfiable.

We denote by $\hookrightarrow^*$ the transitive closure of $\hookrightarrow$, where its path condition %of the
is the conjunction of that of each transition. %along the  transitions are connected by conjunctions, i.e., logical-AND ($\wedge$), resulting
%in the path condition of the symbolic execution.
An \emph{symbolic execution} starting from a symbolic configuration $\SymMPCAngle{p, \alpha,\phi}$ is a sequence of symbolic configurations, written as $\SymMPCAngle{p, \alpha,\phi}\Downarrow
(\alpha',\phi')$,
if $\SymMPCAngle{p, \alpha,\phi}\hookrightarrow^* \SymMPCAngle{{\Pskip}, \alpha',\phi'}$.
Moreover, we denote by $\SymMPCAngle{p, \alpha,\phi}\Downarrow
(\alpha',\phi'):e$
the symbolic execution $\SymMPCAngle{p, \alpha,\phi}\Downarrow
(\alpha',\phi')$ with the symbolic return value $e$.
%
We denote by {\SymExe} the set of all the symbolic executions $\SymMPCAngle{p, \alpha_0,\True}\Downarrow
(\alpha,\phi):e$ of the program $p$.
Note that $\alpha_0$ is the initial symbolic state.
Recall that we assumed all the (concrete) executions always terminate,
thus {\SymExe} is a finite set of finite sequence of symbolic configurations.

\subsection{Relating Symbolic Executions to Concrete Executions}
A symbolic execution $t=\SymMPCAngle{p, \alpha_0,\True}\Downarrow
(\alpha,\phi):e$ represents the set of (concrete) executions
starting from the states $\sigma\in \InitState_0$ such that
$\sigma\models \phi$. Formally, consider $\sigma\in  \InitState_0$ such that
$\sigma\models \phi$, by concretizing all the symbolic values
of variables $x$ in each symbolic state $\alpha'$ with concrete values $\sigma(\alpha'(x))$ and projecting out all the path conditions,
the symbolic execution $t$ is the execution $\MPCAngle{p, \sigma}\Downarrow \sigma': \sigma(e)$, written as
$\sigma(t)$. For the execution $\MPCAngle{p, \sigma}\Downarrow \sigma': v$,
there are a unique symbolic execution $t$ such that $\sigma(t)=\MPCAngle{p, \sigma}\Downarrow \sigma': v$
and a unique execution $\MPCAngle{\widehat{p}_\varrho, \sigma}\Downarrow \sigma': v$ in the program $\widehat{p}_\varrho$.
%Furthermore, there exists one corresponding execution $\MPCAngle{\widehat{p}_\varrho, \sigma}\Downarrow \sigma': v$,
We denote by $\RW_{\varrho,\sigma}(t)$
the execution $\MPCAngle{\widehat{p}_\varrho, \sigma}\Downarrow_\varrho \sigma': v$
and denote by $\RW_{\varrho,\sigma}^{\Pub}(t)$
the sequence $\MPCAngle{\widehat{p}_\varrho, \sigma}\Downarrow_\varrho^{\Pub} \sigma': v$.
%and denoted by ${\tt Stmt}_{\widehat{p}_\varrho}(t)$ the sequence of statements in $\MPCAngle{\widehat{p}_\varrho, \sigma}\Downarrow_\varrho^{\Pub} \sigma': v$.
%Remark that $\MPCAngle{\widehat{p}_\varrho, \sigma}\Downarrow_\varrho^{\Pub} \sigma': v$ for all $\sigma\in \InitState_0$ such that
%$\sigma\models \phi$ have the same sequence of statements.

%
%
%Since the execution $\sigma(\SymMPCAngle{p, \alpha_0,\True}\Downarrow
%(\alpha,\phi):e)$ has a unique corresponding execution $\sigma(\SymMPCAngle{\widehat{p}_\varrho, \alpha_0,\True}\Downarrow
%(\alpha,\phi):e)$ and vice versa,
%Therefore, we can denote by $\RW_{\varrho,\sigma}(\SymMPCAngle{p, \alpha_0,\True}\Downarrow
%(\alpha,\phi):e)$ and $\RW_{\varrho,\sigma}^{\Pub}(\SymMPCAngle{p, \alpha_0,\True}\Downarrow_\varrho^{\Pub}
%(\alpha,\phi):e)$ the executions $\sigma(\SymMPCAngle{\widehat{p}_\varrho, \alpha_0,\True}\Downarrow
%(\alpha,\phi):e)$ and $\sigma(\SymMPCAngle{\widehat{p}_\varrho, \alpha_0,\True}\Downarrow_\varrho^{\Pub}
%(\alpha,\phi):e)$, respectively.



For every adversary-chosen private inputs $\vec{v}_k\in\Dom^k$, result $v\in\Rng$,
and  initial state $\sigma\in\leak_{\tt iw}^p(v,\vec{v}_k)$,
we can reformulate the set $\leak_{\tt rw}^{p,\varrho}(v,\sigma)$ as follows.
(Recall that $\leak_{\tt rw}^{p,\varrho}(v,\sigma)$ is
the set of states $\sigma'\in\leak_{\tt iw}^p(v,\vec{v}_k)$ such that
$\MPCAngle{\widehat{p}_\varrho,\sigma'}\Downarrow_\varrho^{\Pub} \sigma_1':v$
and $\MPCAngle{\widehat{p}_\varrho,\sigma}\Downarrow_\varrho^{\Pub} \sigma_1:v$ are identical.)
%Suppose $t=\SymMPCAngle{p, \alpha_0,\True}\Downarrow
%(\alpha,\phi):e$ is the symbolic execution such that $\RW_{\varrho,\sigma}(t)$ is the execution starting from $\MPCAngle{\widehat{p}_\varrho, \sigma}$.
%This implies that $\sigma\models \phi\wedge e=v$.

\begin{proposition}\label{prop:symbleakage}
For each state $\sigma'\in\leak_{\tt iw}^p(v,\vec{v}_k)$,
$\sigma'\in\leak_{\tt rw}^{p,\varrho}(v,\sigma)$ iff for every symbolic execution $t'=\SymMPCAngle{p, \alpha_0,\True}\Downarrow
(\alpha',\phi'):e'\in \SymExe$ such that $\sigma'\models \phi'\wedge e'=v$,   $\RW_{\varrho,\sigma}^{\Pub}(t)$ and $\RW_{\varrho,\sigma'}^{\Pub}(t')$ are identical,
where $t$ is a symbolic execution $\SymMPCAngle{p, \alpha_0,\True}\Downarrow
(\alpha,\phi):e$ such that $\sigma\models \phi\wedge e=v$.
\end{proposition}

Proposition~\ref{prop:symbleakage} allows to consider only the symbolic executions $\SymMPCAngle{p, \alpha_0,\True}\Downarrow
(\alpha,\phi):e\in \SymExe$ such that $\sigma\models \phi\wedge e=v$ when checking
if $\varrho$ is perfect or not.
%The set $\Sigma^p_{v,v_1,\cdots,v_k}$ can be characterized by the satisfying models of the conjunction of
%the constraints $\phi_i\wedge e_i=v$, where the symbolic executions $\SymMPCAngle{p, \alpha_0,\True}\Downarrow_\varrho^{\Pub}
%(\alpha_i,\phi_i):e_i$ have the same observable executions $\RW_\sigma(\SymMPCAngle{p, \alpha_0,\True}\Downarrow_\varrho^{\Pub}
%(\alpha_i,\phi_i):e_i)$ for every $\sigma\in \Sigma^p_{v,v_1,\cdots,v_k}$.


%Moreover, we observe that the equivalence of $\RW_\sigma(\SymMPCAngle{p, \alpha_0,\True}\Downarrow_\varrho^{\Pub}
%(\alpha,\phi):e)$ and $\RW_{\sigma'}(\SymMPCAngle{p, \alpha_0,\True}\Downarrow_\varrho^{\Pub}
%(\alpha',\phi'):e')$ can be represented by
%a constraint, ensuring that for each pair of corresponding symbolic states
%$(\alpha,\alpha')$ in two symbolic executions,
%for each public variable $x$, the symbolic values $\alpha(x)$
%and $\alpha'(x)$ are equivalent under the assignments $\sigma$ and $\sigma'$, respectively.
%Formally, $\Sigma^{p,\varrho}_{v,\sigma}$ is the set of satisfying models
%of the following logical constraint:
%
%\begin{align*}
%& \bigwedge_{i=1}^k x_i=v_i  \wedge \forall \SymMPCAngle{p, \alpha_0,\True}\Downarrow
%(\alpha',\phi'):e'\in \SymExe. \phi'\wedge e'=v \Rightarrow \\
% & |\sigma(\SymMPCAngle{p, \alpha_0,\True}\Downarrow
%(\alpha,\phi):e)|=|\SymMPCAngle{p, \alpha_0,\True}\Downarrow
%(\alpha',\phi'):e'|\\
%\end{align*}
%\[\wedge \bigwedge_j  \]
%
% Let $f$ be the initial state that
%maps private input variables $x_i$ to their primed versions $x_i'$.
%Then, $\RW_\sigma(\SymMPCAngle{p, \alpha_0,\True}\Downarrow_\varrho^{\Pub}
%(\alpha,\phi):e)$ and $\RW_{\sigma'}(\SymMPCAngle{p, \alpha_0,\True}\Downarrow_\varrho^{\Pub}
%(\alpha',\phi'):e')$ are equivalent, denoted by $\SymMPCAngle{p, \alpha_0,\True}\Downarrow
%(\alpha,\phi):e\simeq_{\Pub}\SymMPCAngle{p, \alpha_0,\True}\Downarrow
%(\alpha',\phi'):e'$ iff
%\[|\SymMPCAngle{p, \alpha_0,\True}\Downarrow
%(\alpha,\phi):e|=|\SymMPCAngle{p, \alpha_0,\True}\Downarrow
%(\alpha',\phi'):e'|\wedge \bigwedge_j \]
%
%
% , ensuring that for each pair of corresponding symbolic states
%$(\alpha,\alpha')$ in two symbolic executions,
%for each public variable $x$, the symbolic values $\alpha(x)$
%and $\alpha'(x)$ are equivalent under the assignments $\sigma$ and $\sigma'$, respectively.
%Therefore, $\Sigma^{p,\varrho}_{v,\sigma}$ is the same as the following set:
%\[\{\sigma\in \Sigma^p_{v,v_1,\cdots,v_k}\mid \} \]
%
%\[\bigwedge_{\SymMPCAngle{p, \alpha_0,\True}\Downarrow
%(\alpha,\phi):e\in \SymExe}  (\phi\wedge e=v) \]
%
%
%In the next section, we will leverage this observation to identify secret variables
%that cannot declassified.


\subsection{Reasoning about Symbolic Executions}
We leverage Proposition~\ref{prop:symbleakage} to identify secret variables
that can be declassified  without compromising the security by reasoning about symbolic executions. %Instead of checking all the possible variables,
%we will check secret conditions in the least IFP $\varrho$ obtained by our type system
%by reasoning about symbolic executions.
%
%We start by introducing some notations.
%Let us fix a secret condition $x$ of ${\Pif} \ x \ {\Pthen} \  p' \ {\Pelse} \ p''$ and a secure IFP $\varrho$.
For each expression $\phi\in \Exp$,
$\Primed(\phi)$ denotes the ``primed" expression $\phi$ where each  private input variable $x_i$
is replaced by $x_i'$ (i.e.,  its primed version).

Consider two symbolic executions
$t=\SymMPCAngle{p, \alpha_0,\True}\Downarrow
(\alpha,\phi):e$ and $t'=\SymMPCAngle{p, \alpha_0,\True}\Downarrow
(\alpha',\phi'):e'$. Assume ${\Pif} \ x \ {\Pthen} \  p' \ {\Pelse} \ p''$ is not used with any secret-dependent conditions.
Recall that we assumed  $x$ is used only in ${\Pif} \ x \ {\Pthen} \  p' \ {\Pelse} \ p''$.
Then, $t$ and $t'$ execute the same subsequence (say $p_1,\cdots,p_m$) of the statements that are ${\Pif} \ x \ {\Pthen} \  p' \ {\Pelse} \ p''$.
Let $e_1,\cdots,e_m$ (resp. $e_1',\cdots,e_m'$) be symbolic values of $x$
when executing $p_1,\cdots,p_m$ in the symbolic execution $t$ (resp. $t'$).
Define the constraint $\Psi_x(t,t')$ as
\begin{center}
 $\Psi_x(t,t') \triangleq\big(\phi\wedge \Primed(\phi')\wedge e=\Primed(e')\big)\Rightarrow \big(\bigwedge_{i=1}^m e_i=\Primed(e_i')\big)$
\end{center}

%
%added into the path condition $\phi$
%(resp. $\phi'$) for the statement ${\Pif} \ x \ {\Pthen} \  p' \ {\Pelse} \ p''$.
%For instance, if $x$ has symbolic value $\alpha(x)$ and the symbolic execution goes to the then-branching
%(resp. else-branching), then $\alpha(x)$  (resp. $\neg\alpha(x)$) is added into the path condition.
%Define the constraint $\Psi_x(t,t')$ as:
%%
%\begin{center}
%$\Psi_x(t,t') \triangleq\big(\phi\wedge \Primed(\phi')\wedge e=\Primed(e')\big)\Rightarrow \big(\bigwedge_{i=1}^m \Primed(e_i)\wedge \bigwedge_{i=1}^h e_i'\big).$
%\end{center}
%
Intuitively, $\Psi_x(t,t')$ asserts that for every pair of
states $\sigma,\sigma'\in \InitState_0$
if  $\sigma$ (resp. $\sigma'$ ) satisfies the path condition $\phi$ (resp. $\phi'$), $\sigma(e)$
and $\sigma'(e')$ are identical, then for each $1\leq i\leq m$,  
the values of $x$ are the same when executing the conditional statement $p_i$ in both $\RW_{\varrho,\sigma}(t)$ and $\RW_{\varrho,\sigma'}(t')$.


%Intuitively, $\Psi_x(t,t')$ asserts that for every pair of
%initial states $\sigma,\sigma'\in \InitState_0$
%if $\sigma$ (resp. $\sigma'$ ) satisfies the path condition $\phi$ (resp. $\phi'$),
%$\sigma(e)$ and $\sigma'(e')$ are identical,
%then
%$e_i$ for all $1\leq i\leq m$ are true under the initial state $\sigma'$
%and $e_i'$ for all $1\leq i\leq h$ are true under the initial state $\sigma$.
%Note that $e_i$ for all $1\leq i\leq m$ (resp. $e_i'$ for all $1\leq i\leq h$) are already true under the initial state $\sigma$
%(resp. $\sigma'$).

%
% such that ${\tt Stmt}_{\widehat{p}_\varrho}(t)$
%and ${\tt Stmt}_{\widehat{p}_\varrho}(t')$ are the same,
%namely, $\RW_{\varrho,\sigma}^{\Pub}(t)$ and $\RW_{\varrho,\sigma'}^{\Pub}(t')$
%have the same sequence of statements for all $\sigma,\sigma'\in \leak_{\tt iw}^p(v,v_1,\cdots,v_k)$
%such that $\sigma\models \phi$  and $\sigma\models \phi'$.
%Since $x$ has type $\Sec$ in $\varrho$, the
%statement ${\Pif} \ x \ {\Pthen} \  p' \ {\Pelse} \ p''$
%must be replaced by $\Enc_\varrho(c\& x, p');\Enc_\varrho(c\& \neg x, p'')$ in
%the program $\widehat{p}_\varrho$ for some Boolean expression $c$.
%This implies that $t$ and $t$ have the same subsequence of statements $p_1,\cdots,p_m$
%that are ${\Pif} \ x \ {\Pthen} \  p' \ {\Pelse} \ p''$.
%Let  $e_1,\cdots,e_m$ (resp. $e_1',\cdots,e_m'$) be the symbolic values
%of $x$ in the corresponding symbolic states $\alpha_{1},\cdots,\alpha_{m}$ (resp. $\alpha_{1}',\cdots,\alpha_{m}'$) of $p_1,\cdots,p_m$ in the symbolic execution $t$
%(resp. $t'$).
%Define the constraint $\Psi_x(t,t')$ as:
%%
%\begin{center}
%$\Psi_x(t,t') \triangleq\big(\phi\wedge \Primed(\phi')\wedge e=\Primed(e')\big)\Rightarrow \bigwedge_{i=1}^m e_i=\Primed(e_i').$
%\end{center}
%%
%Intuitively, $\Psi_x(t,t')$ asserts that for every pair of
%initial states $\sigma,\sigma'\in \InitState_0$
%such that $\sigma$ (resp. $\sigma'$ ) satisfies the path condition $\phi$ (resp. $\phi'$), $\sigma(e)$
%and $\sigma'(e')$ are identical, we have: for each $1\leq i\leq m$, the secret condition $x$
%at the conditional statement $p_i$ has the same value in the symbolic states $\alpha_{i}$
%and $\alpha_{i}'$.

\begin{proposition}\label{prop:symbleakage2}
For each pair of
states $\sigma,\sigma'\in\leak_{\tt iw}^p(v,\vec{v}_k)$ such that
$\sigma\models \phi\wedge e=v$ and $\sigma'\models \phi'\wedge e'=v$,
if $\Psi_x(t,t')$ is valid and $\RW_{\varrho,\sigma}^{\Pub}(t)$ and $\RW_{\varrho,\sigma'}^{\Pub}(t')$ are identical,
then $\RW_{\varrho',\sigma}^{\Pub}(t)$ and $\RW_{\varrho',\sigma'}^{\Pub}(t')$ are identical,
where $\varrho'=\varrho[x\mapsto \Pub]$.
\end{proposition}

Recall that  $x$ can be declassified in a perfect security policy $\varrho$
if $\varrho'=\varrho[x\mapsto \Pub]$ is still perfect, namely, $\MPCAngle{\widehat{p}_{\varrho'},\sigma'}\Downarrow_{\varrho'}^{\Pub} \sigma_1':v$
and $\MPCAngle{\widehat{p}_{\varrho'},\sigma}\Downarrow_{\varrho'}^{\Pub} \sigma_1:v$ are identical
for every adversary-chosen private inputs $\vec{v}_k\in\Dom^k$, result $v\in\Rng$,
and states $\sigma,\sigma'\in\leak_{\tt iw}^p(v,\vec{v}_k)$.
By Proposition~\ref{prop:symbleakage2},
if  $\Psi_x(t,t')$ is valid for each pair of symbolic executions $t,t'\in\SymExe$,
we can deduce that $\varrho'$ is still perfect.

\begin{theorem}\label{thm:mainresult}
If $\varrho\models_p f(\vec{x})$ and $\Psi_x(t,t')$ is valid for each pair of symbolic executions $t,t'\in\SymExe$,  %such that ${\tt Stmt}_{\widehat{p}_\varrho}(t)$
%and ${\tt Stmt}_{\widehat{p}_\varrho}(t')$ are identical,
then $\varrho[x\mapsto \Pub]\models_p f(\vec{x})$.
\end{theorem}

\begin{example}
Consider two symbolic executions $t$ and $t'$ in the motivating example
such that the path condition $\phi$ (resp. $\phi'$) of $t$ (resp. $t'$)
is ${\tt a}\geq{\tt b}\wedge {\tt c}>{\tt a}$ (resp. ${\tt a}<{\tt b}\wedge {\tt c}>{\tt b}$),
and both return the result 3.
The secret branching variable {\tt c2} has the symbolic values ${\tt c}>{\tt a}$ (resp. ${\tt c}>{\tt b}$)
in $t$ and $t'$, respectively.
%For the secret condition {\tt c2},  ${\tt c}>{\tt a}$ (resp. ${\tt c}>{\tt b}$) is added into $\phi$
%(resp. $\phi'$).
Then
\[\Psi_{\tt c2}(t,t')=\left\{\begin{array}{c}
  \big(({\tt a}\geq{\tt b}\wedge {\tt c}>{\tt a})\wedge\\
  ({\tt a}'<{\tt b}'\wedge {\tt c}'>{\tt b}')\wedge 3 =3\big)\\ \Rightarrow
  \big(({\tt c}>{\tt a})= ({\tt c}'>{\tt b}')\big)\\
\end{array}\right\}.\]
Obviously, $\Psi_{\tt c2}(t,t')$ is valid.
We can show that for any other pair $(t,t')$ of symbolic executions,
$\Psi_{\tt c2}(t,t')$ is always valid.
Therefore, the secret branching variable {\tt c2} can be declassified in any perfect security policy $\varrho$.

In contrast, the secret branching variable {\tt c1} has the symbolic value ${\tt a}<{\tt b}$
in both $t$ and $t'$. Then,
\[\Psi_{\tt c1}(t,t')=\left\{\begin{array}{c}
  \big(({\tt a}\geq{\tt b}\wedge {\tt c}>{\tt a})\wedge\\
  ({\tt a}'<{\tt b}'\wedge {\tt c}'>{\tt b}')\wedge 3 =3\big)\\ \Rightarrow
  \big(({\tt a}<{\tt b})= ({\tt a}'<{\tt b}')\big)\\
\end{array}\right\}.\]
$\Psi_{\tt c1}(t,t')$ is not valid, thus the secret branching variable {\tt c1} cannot be declassified.
\end{example}

\noindent
{\bf Refinement}.
Theorem~\ref{thm:mainresult} allows us to
check if the secret branching variable $x$ of a conditional statement ${\Pif} \ x \ {\Pthen} \  p' \ {\Pelse} \ p''$ that does not used with
any secret-dependent conditions can be declassified.
%
After that, if $x$ can be declassified without compromising the security,
we feed back the result to the type system before checking the next secret branching variable.
This allows us to refine the security level of variables that are updated in branches, namely,
the type inference rule {\sc T-If} is refined into the following one:

\begin{center}
$\dfrac{\makecell[c]{\cxt'= (\mbox{can $x$ be declassified } \ ? \ \Pub:\varrho(x)) \\ \cxt\sqcup \cxt'\vdash p_1:\varrho\Rightarrow\varrho_1 \quad \cxt\sqcup \cxt' \vdash p_2:\varrho\Rightarrow\varrho_2 \quad\varrho'=\varrho_1 \sqcup\varrho_2 }}{\cxt \vdash {\Pif} \ x \ {\Pthen} \  p_1 \ {\Pelse} \ p_2:\varrho\Rightarrow\varrho'}$[{\sc T-If}]
\end{center}
