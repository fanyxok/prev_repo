\begin{abstract}
Secure multi-party computation (MPC) is a promising technique for privacy-persevering applications.
Thus, a large number of MPC frameworks have been created to reduce the burden of designing customized
protocols and allow non-experts to quickly develop and deploy MPC applications.
To improve performance, recent MPC frameworks allow users to declare secret variables so that only the values of secret variables are to
be protected. However, in practice, it could be highly non-trivial for non-experts to
specify secret variables properly: declaring too many secret variables degrades
the performance while declaring too less secret variables compromises security.
To address this problem, in this work, we propose an automated security policy synthesis
to declare as few secret variables as possible but without compromising security.
Our approach is a synergistic integration of a type inference system and a symbolic reasoning approach, where the former 
is able to quickly infer a sound security policy and the latter allows to identify 
secret variables in a security policy that can be declassified without compromising security. Moreover, the results from the symbolic reasoning approach is fed back to
type inference to refine security types further. We implement our approach in a tool, named \TNAME.
Experimental results on five typical MPC applications demonstrates the efficacy of our approach.
\end{abstract}
