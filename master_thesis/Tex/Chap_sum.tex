\chapter{总结与展望}
多方安全计算协议经过不断研究诞生了如GMW协议、BMR协议、PSSZ协议等一系列具有较高效率的实用协议,由于多方安全计算的实际需求也出现了Sharemind、Obliv-C、ABY等接近十种程序开发框架。形式化的验证多方计算程序和框架对提高安全性有重要的意义。本文基于对多方安全计算领域特定语言Wys*的分析,观察Wys*在形式化验证自身时遗留下的问题,设计和实现了一种基于类型系统的在Wys*进行电路翻译前验证待翻译的程序语义是否符合Wys*翻译能力的方法,作为Wys*验证链的补充。其他验证工作如SecreC也有相似的问题,因此本文的方法也为解决这一类问题提供了思路。根据文献调查,目前对多方安全计算程序进行形式化验证的工作只有有限的几篇文献\citep{almeida2018enforcing,rastogi2019textsc,rastogi2017wys}。多方安全计算程序区别与一般的逻辑程序,涉及网络、电路、安全协议等方方面面,对多方安全计算程序和框架进行完整的形式化验证仍有极大的研究空间。
基于本文设计的验证方法和研究过程中对多方安全计算程序验证工作的分析,后续的研究工作可以从以下方向进行:
\begin{itemize}
\item 设计具有完整的隐私数据信息流安全性验证和程序正确性验证的多方安全计算程序验证方案,目前研究领域的验证工作中只关注隐私数据的安全性或者程序正确性而没有进行结合,丧失了整体的可靠性。
\item 对一个布尔电路翻译模块进行形式化验证,以Wys*所使用的翻译算法为例\citep{choi2012secure},先验证该翻译算法的正确性,然后基于该算法的某一主要实现验证该实现的正确性。
\item 验证一个多方安全计算协议程序实现的正确性,这是一个具有挑战性的研究方向,目前研究领域还没有与该方向直接相关的成果。
\item 证明Wys*的底层语义的安全性,这一部分内容较多的涉及Wys*所依赖的Fstar,需要证明Wys*与Fstar的核心语义具有完整的对应关系。
\end{itemize}

