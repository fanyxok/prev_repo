%\chapter{作者简历及攻读学位期间发表的学术论文与研究成果}
%
%\textbf{本科生无需此部分}。
%
%\section*{作者简历}
%
%\subsection*{casthesis作者}
%
%吴凌云,福建省屏南县人,中国科学院数学与系统科学研究院博士研究生。
%
%\subsection*{ucasthesis作者}
%
%莫晃锐,湖南省湘潭县人,中国科学院力学研究所硕士研究生。
%
%\section*{已发表(或正式接受)的学术论文:}
%
%[1] ucasthesis: A LaTeX Thesis Template for the University of Chinese Academy of Sciences, 2014.
%
%\section*{申请或已获得的专利:}
%
%(无专利时此项不必列出)
%
%\section*{参加的研究项目及获奖情况:}
%
%可以随意添加新的条目或是结构。

\chapter[致谢]{致\quad 谢}\chaptermark{致\quad 谢}% syntax: \chapter[目录]{标题}\chaptermark{页眉}
%\thispagestyle{noheaderstyle}% 如果需要移除当前页的页眉
%\pagestyle{noheaderstyle}% 如果需要移除整章的页眉
%\pagestyle{mainmatterstyle} % 与前文页眉页脚格式相同
本论文是在我的导师宋富教授的悉心指导之下完成的。本论文在选题、调研和确定具体内容等诸多方面都受到了宋老师的悉心帮助。本科时学习的宋老师授课的编译原理课程对本文设计研究方法和推进程序实现起到了重要的作用。宋老师踏实严谨的治学态度引导我步入学术研究生涯,令我终生受益。在此我对宋老师表示由衷的感谢!

我要感谢我的家人!感谢父母对我的支持,在生活和学习上都给予了无微不至的帮助,让我在疫情期间依然能将精神集中于本论文的设计撰写。

感谢我的学长学姐、同门的师兄弟以及各位教授,有你们的帮助和教育我才有能力不断的解决困难。我们在上科大度过的学习生活时光会成为我生命中宝贵的记忆。感谢我的同学与我分享快乐,也陪我经历磨难。感谢我的室友一直以来的照顾。

最后我要感谢母校上海科技大学!立志,成才,报国,裕民八字校训每次思考都让我有不同的体悟。感谢信息学院的老师们在学习、生活上提供的指导。感谢四年来我在上科大求学生涯的每一位参与者!

\cleardoublepage[plain]% 让文档总是结束于偶数页,可根据需要设定页眉页脚样式,如 [noheaderstyle]

